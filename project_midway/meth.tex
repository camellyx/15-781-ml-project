\section{Methodology}
\label{sec:meth}

%Data set (5 points): Is the data set completely ready (in-hand, loaded onto a machine, analyzed for outliers or other problems)? Have some main features of the data set been characterized (either visually or statistically)?


\subsection{MNIST}
MNIST is a data set of handwritten digits. It contains a training set of
60,000 examples and a test set of 10,000 examples. The digits have been
size-normalized and centered in a fixed-size image.

We used this data set in our first step experiments. (See Experiments for
more details.)

\subsection{CIFAR-10}
CIFAR-10 is a data set of 60000 32x32 colour images in 10 classes, with
6000 images per class. There are 50000 training images and 10000 test
images. The dataset is divided into five training batches and one test
batch, each with 10000 images. The test batch contains exactly 1000
randomly-selected images from each class. The training batches contain the
remaining images in random order, but some training batches may contain
more images from one class than another. Between them, the training
batches contain exactly 5000 images from each class.

We have succesfully loaded this dataset, and we are working on
fine-tuning parameters of the model to better fit this dataset.

\subsection{Other data sets}
Besides the above two datasets, we will conduct experiements on the
following datasets as well:
\begin{itemize}
\item CIFAR-100 \\
\item Street View House Numbers(SVHN)\\
\item
\end{itemize}

