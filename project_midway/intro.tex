\section{Introduction}
\label{sec:intro}

% Question and Goals (10 points): State the updated version of the research
% question, based on results so far. Is it well-formed and interesting? Are
% there well-defined metrics for success? Is the minimum goal achievable in the
% remaining time? Are the stretch goals interesting?

{\bf Problem:}
Large-scale deep neutral network models have become increasingly popular
to solve hard classification problems and have demonstrated significant
improvements in accuracy. Due to the scale of the neutral network and the
scale of the input data set, performance, in addition to accuracy, has
become a significant factor in such deep learning implementations.

Recent works~\cite{dean2012large, chilimbi14adam} on large-scale machine
learning systems proposes to significantly improve the performance by relaxing
the consistency when updating the weights of each individual unit (e.g., one
weight update can get overwritten by another). One interesting observation in
this paper, alongside the above one, is that this relaxation surprisingly
improves the accuracy of the model.

The hypothesis of this observation (in~\cite{chilimbi14adam}) is that relaxing
consistency introduces some stochastic noise into the training data. This
implicitly mitigates over-fitting of the model and generalizes the resulting
model to classify the test data better.

{\bf Goal:}
In this project, however, we intend to validate and generalize this observation
to other types of noise, which may be introduced by other factors such as lossy
compression, lossy computation, or transient floating point
% TODO: add citation for these
computation/memory/storage errors. We expect our results to pave the way for
future exploration of hardware approximate computation/storage techniques to
accelerate and improve the accuracy for large-scale deep learning algorithms.

More specifically, we intend to answer at least the following new research questions:
\begin{itemize}
  \item What are the points of interest in different neural network models in
    which we can introduce noise? How do we inject errors to emulate these
    noise? How does each model react to these errors? Does the noise improve or
    reduce accuracy of the model when they are introduced at different places?
  \item What are the 
\end{itemize}

{\bf Metrics:} %%% Not sure about this part %%%
We are interested in the improvement of error rate, convergence rate
and training speed.
We consider a 0.99\% decrease in test error rate to be significant.




