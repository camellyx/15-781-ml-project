\section{Introduction}
\label{sec:intro}

Large-scale deep neutral network models have become increasingly popular to
solve hard classification problems and have demonstrated significant
improvements in accuracy. Compared to traditional statistical machine learning
methods, which require a human domain expert that can construct a good set of
features as input dataset, deep learning models do not require a hand crafted
feature set to begin with, and hence is more powerful and suitable for hard AI
tasks such as speech recognition or visual object classification. Without any
hand crafting of the raw input data, deep neural network machine learning models
can learn a hierarchy of features by itself in the first several layers of
neural network model. Then, in the deepest layer of the model, a set of features
are selected and weighted for each output to generate a prediction. Avoiding the
inevitable human error in feature selection, deep learning often outperforms
traditional approaches in those hard classification problems in terms of
accuracy.

In order to train a more complicated model which includes feature selection
capability, deep neural network is typically trained with more data than a
traditional machine learning method. Due to the scale of the deep neutral
network (with multiple layers of neurons) and the scale of the input data set,
the performance of these models, in addition to accuracy, has become a
significant factor in such implementations. Recent
work~\cite{chilimbi14adam,wan2013dropconnect} on large-scale machine learning
systems propose to significantly improve the performance by relaxing the
consistency when training neural network models (e.g.,  weights will not be
updated in each iteration). One interesting observation in these papers,
alongside the above one, is that relaxation surprisingly improves the {\em
accuracy} of the deep learning model on the test data.

However, the effect of noise on deep learning models has never been
systematically studied, nor is the underlying reason for the improved accuracy.
One hypothesis of the above observation is that relaxing consistency introduces
stochastic noise into training process~\cite{chilimbi14adam}.  This implicitly
mitigates over-fitting of the model and generalizes the model better to classify
test data. Another hypothesis of this observation is that the introduced noise
eliminates the memorization effect of a deep neural network, and hence allows
the model to capture the general observation of the training data that can be
applied to the test data well.

Our work, taking previous works as examples and guidance, tries to
systematically study the effect of introducing {\em different noise} into {\em
different components} of {\em different types} of deep learning neural networks.
We observe that a reasonable amount, and a reasonable magnitude of noise, when
introduced into a deep learning model, can improve the accuracy and the
convergence rate of the model. We hope that our work can provide insights into
future methods of approximate deep learning model, and inspire and motivate more
work to take advantage of the 

