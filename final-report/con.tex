\section{Conclusions}
\label{sec:conclusions}
In this project, we systematically perform experiments on studying the
effect of adding noise into deep learning neural networks.
We conduct experiments on adding different noise into different components
of neural network models. The experiment results show that adding noise almost always improves accuracy.
Our main observations are:
(1) Noise added during early stage of the model can be better integrated
while noise added during late stage of the model tends to cause
fluctuation of accuracy;
(2) Complicated neural network models can integrate and absorb
noise better than simple neural network models;
(3) Sometimes adding noise can improve not only accuracy but also
convergence rate.
We hope that this experimental study can provide insights into future design
of deep learning neural network models and machine learning hardwares.

Beyond this project, we hope to pursue three major research directions: (1)
Conduct more thorough experiments that quantitatively analyze the effect of
noise on deep learning models; (2) Provide theoretical explanation for the
effect of noise on deep learning models based on our experiment results and
findings; (3) Design and explore more efficient computer hardware and systems
for deep learning models.
